% Source: https://github.com/koosaga/DeobureoMinkyuParty/blob/master/teamnote.tex

\begin{misc}{Tutte matrix (perfect matching test)}
	\begin{equation*}
		M_{ij} =
			\begin{cases}
				x_{ij} & \text{if } ij \in E, i < j \\
				-x_{ji} & \text{if } ij \in E, i > j \\
				0 & \text{otherwise}
			\end{cases}
	\end{equation*}
	$\det(M) = 0 \iff $ no perfect matching w.h.p. \\
\end{misc}

\begin{misc}{Kirchhoff's theorem (\# of spanning trees)}
	\begin{equation*}
		M_{ij} =
			\begin{cases}
				\text{deg}_{\text{in}}(i) & \text{if } i = j \\
				-\#(ij\text{ edges}) & \text{if } i \neq j
			\end{cases}
	\end{equation*}
	$M' = M$ with $i$-th row and column removed \\
	$\det(M') = $ \# of oriented spanning trees rooted at $i$
\end{misc}

\begin{misc}{Cayley's formula (\# of labelled trees)}
	For degree sequence $d_1, ..., d_n$:
	\begin{equation*}
		\frac{(n-2)!}{(d_1 - 1)!(d_2 - 1)! \cdots (d_n - 1)!}
	\end{equation*}
	$n_1n_2...n_kn^{k-2} =$ for $k$ existing trees of size $n_i$ \\
	$kn^{n-k-1} =$ forests on $n$ vertices with $k$ components such that $1, ..., k$ belong to different components \\
	$x_1 \ldots x_n(x_1+ \ldots +x_n)^{n-2} = \sum_T x_1^{d_1(T)} \ldots x_n^{d_n(T)}$
\end{misc}

\begin{misc}{Pick's theorem}
	For a simple polygon with integer vertices, area $A$, $i$ grid points in the interior, and $b$ grid points on the boundary:
	$A=i+b/2-1$.
\end{misc}

\begin{misc}{\# of partitions into positive integers}
	\begin{align*}
		p(0) &= 1 \\
		p(n) &= \sum_{k \in \mathbb{Z} \setminus \{0\}} (-1)^{k+1} p(n - k(3k-1)/2)
	\end{align*}
\end{misc}

\begin{misc}{Burnside's lemma}
	$G =$ group that acts on a set $X$ \\
	$X^g =$ set of elements fixed by $g \in G$ \\
	$X/G =$ set of orbits, i.e. equivalence classes by $G$
	\begin{equation*}
		|X/G| = \frac{1}{|G|} \sum_{g \in G} |X^g|
	\end{equation*}
\end{misc}

\begin{misc}{Sums}
	\begin{align*}
		c^a + \ldots + c^b &= \frac{c^{b+1} - c^a}{c-1} \text{~~~~~if } c \neq 1 \\
		1 + \ldots + n &= \frac{n(n+1)}{2} \\
		1^2 + \ldots + n^2 &= \frac{n(2n+1)(n+1)}{6} \\
		1^3 + \ldots + n^3 &= \frac{n^2(n+1)^2}{4} \\
		1^4 + \ldots + n^4 &= \frac{n(n+1)(2n+1)(3n^2+3n-1)}{30} \\
	\end{align*}
\end{misc}

% \begin{misc}{Power series}
% 	\begin{align*}
% 		f(x) = \sum_{n \geq 0} \frac{f^{(n)}(a)}{n!}(x-a)^n &&
% 		e^x = \sum_{n \geq 0} \frac{x^n}{n!}
% 	\end{align*}
% 	\begin{align*}
% 		\ln(1+x) = \sum_{n \geq 1} \frac{(-1)^{n+1}x^n}{n} &&
% 	\end{align*}
% \end{misc}

% TODO: Rewrite rest of old appendix
